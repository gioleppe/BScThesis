
\begin{singlespace}
	\begin{center}
		\textsc{riassunto}
	\end{center}

I sistemi di Mobile CrowdSensing permettono di creare data set utilizzando le rilevazioni prodotte dai sensori presenti su dispositivi personali, indossabili, etc.
Uno degli aspetti di interesse per tale paradigma è misurare la qualità, ovvero la rappresentatività, dei dati che possono essere acquisiti. Tale problema prende il nome di Data Coverage.
	
In questa tesi ci prefiggiamo di implementare e applicare un modello probabilistico di Data Coverage in un contesto urbano basandoci sullo stato dell'arte.
Applichiamo il modello al dataset Microsoft Geolife, dopo averlo opportunamente arricchito per ovviare alla mancanza di dati.
Mettiamo inoltre a confronto il data set ottenuto con l'originale per valutare la bontà dell'arricchimento effettuato.
Nel seguito, studiamo la Data Coverage calcolata dal modello per una serie di scenari di Mobile Crowdsensing nell'area metropolitana di Pechino.
Gli scenari selezionati si legano a specifiche problematiche proprie delle smart cities, siano esse esigenze di monitoraggio ambientale, controllo dei flussi di traffico o di raccolta dati da punti di interesse dislocati nella metropoli.
Per ciascuno scenario mostriamo i risultati del nostro esperimento per mezzo di opportune mappe di calore.
	
Tramite questo studio otteniamo alcuni data set di Coverage per ciascuno scenario considerato, adatti a pianificare future campagne di raccolta dati di vario genere.
	
\end{singlespace}
